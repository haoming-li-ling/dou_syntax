\documentclass[11pt]{article}
\usepackage[letterpaper, margin=1in]{geometry}
% \usepackage[skip=\baselineskip, tocskip=0pt]{parskip}
\usepackage{amsmath, amsfonts, amssymb, amsthm, mathtools}
\usepackage{stmaryrd}
\usepackage{fontspec}
\let\latextextsubscript\textsubscript
\AtBeginDocument{
  \let\textsubscript\latextextsubscript
  \newcommand{\sub}[1]{\textsubscript{#1}}
  \newcommand{\gap}[1]{\rule{1em}{0.4pt}\textsubscript{#1}}
}
\let\latextextsuperscript\textsuperscript
\AtBeginDocument{
  \let\textsuperscript\latextextsuperscript
}
% \usepackage[largesc]{newpxtext}
% \usepackage[vvarbb]{newpxmath}
\usepackage{newtx}
\usepackage{csquotes}
\usepackage{markdown}

\usepackage{pifont}
% \usepackage{txfonts}
\let\eachwordone=\rmfamily
\let\eachwordtwo=\rmfamily
\let\eachwordthree=\rmfamily
\let\rm=\relax
% \usepackage{linguex, cgloss}
\usepackage[linguex]{expex-glossonly}
\lingset{everygla={\itshape}}
\renewcommand{\firstrefdash}{}
\usepackage[hidelinks]{hyperref}
\usepackage{cleveref}
\crefname{ExNo}{}{}
\crefname{SubExNo}{}{}
\renewcommand{\theExNo}{\arabic{ExNo}}
\renewcommand{\theSubExNo}{\theExNo\alph{SubExNo}}
\creflabelformat{SubExNo}{(#2#1#3)}
\creflabelformat{ExNo}{(#2#1#3)}
\crefrangelabelformat{SubExNo}{(#3#1#4--#5\crefstripprefix{#1}{#2}#6)}
\crefrangelabelformat{ExNo}{(#3#1#4)--(#5#2#6)}
\usepackage{enumitem}
\usepackage[normalem]{ulem}
\usepackage{xcolor}
\setlength{\parskip}{1ex}
\setlength{\parindent}{0pt}
\setlist[1]{left=0pt}
% \renewcommand\thesection{\Roman{section}.}
% \renewcommand\thesubsection{\arabic{subsection}.}
% \renewcommand\thesubsubsection{\Alph{subsubsection}}

\usepackage[
  backend=biber,
  natbib=true,
  style=unified,
  maxcitenames=3,
  maxbibnames=99
]{biblatex}
\addbibresource{../Chinese_Modals.bib}
\addbibresource{../Distributivity.bib}
\AtBeginDocument{
  \setlength{\Extopsep}{0\baselineskip}
  \setlength{\Exredux}{0\baselineskip}
  \settowidth{\Exlabelwidth}{(00)}
  % \setlength{\Exlabelwidth}{.4\Exlabelwidth}
  \setlength{\Exlabelsep}{.4\Exlabelsep}
  % \setlength{\SubExleftmargin}{\SubExleftmargin}
  % \setlength{\SubSubExleftmargin}{\SubSubExleftmargin}
}

\newcommand{\II}{\I{[}}
\newcommand{\A}{\(\overline{\text{A}}\)}
\usepackage{tikz}
\usepackage[linguistics]{forest}
\usetikzlibrary{positioning}
\usetikzlibrary{arrows.meta}
\usetikzlibrary{tikzmark}
\usepackage{tikz-cd}
\tikzcdset{
  arrow style=math font,
  % diagrams={>={Straight Barb[scale=0.8]}}
}

\tikzset{every label/.style={font=\footnotesize}}

\forestset{
  default preamble={
    for tree={
      inner sep=0pt,
      % draw,
    }
  },
  great empty nodes/.style={
    for tree={
      calign=fixed edge angles,
      calign primary angle=-60,
      calign secondary angle=60,
    l=7mm},
    delay={
      where content={}{
        shape=coordinate,
    for current and siblings={anchor=north}}{}}
  },
  downroof/.style={
    for children={
      if n=1{
        edge path'={
          (.parent first) -- (!u.parent anchor) -- (!ul.parent last) -- cycle
        }
      }{no edge}
    }
  }
}
\usepackage{tabularx}
\usepackage{booktabs}
\renewcommand\tabularxcolumn[1]{m{#1}}% for vertical centering text in X column

\AtBeginDocument{% to do this after unicode-math has done its work
  \renewcommand{\setminus}{\mathbin{\backslash}}%
}
\usepackage{paracol}
\globalcounter{ExNo}
\usepackage{titlesec}
\titlespacing{\paragraph}{0pt}{0pt}{1em}
\setlength{\parskip}{.1ex}
% \setlength{\parskip}{1ex}
\setlength{\parindent}{0pt}

\title{Arguments for a bi-clausal analysis of Mandarin deontic modals}
\author{Haoming Li}

\DeclareMathOperator{\atom}{\textsc{Atom}}
\DeclareMathOperator{\alt}{\textsc{Alt}}
\DeclareMathOperator{\xh}{\textsc{Exh}}
\newcommand{\exh}{\ensuremath{\xh}}
\newcommand{\Exh}{\ensuremath{\mathcal{E}\mathit{xh}}}
\newcommand{\Pex}{\ensuremath{\mathcal{P}\mathit{ex}}}
\newcommand{\F}{\ensuremath{_{\mathrm{F}}}}
\DeclareMathOperator{\prt}{Part}
\DeclareMathOperator{\dom}{dom}
\newcommand{\opfs}{Op\(_{\mathrm{FS}}\)}
\newcommand{\topobj}{Topic\(_{\mathrm{int}}\)P}
\newcommand{\vmod}{V\(_{\mathrm{Mod}}\)}

\begin{document}
% \maketitle
\strut \hfill \textbf{Diagnosing modal clause structure with focus-sensitive operators in Mandarin Chinese} \hfill  \strut

\paragraph{Introduction}
\label{sec:introduction}

There is a long debate on the nature of the clausal spine in modalized sentences in Mandarin: 
are modal verbs clause-embedding (bi-clausal, \Next, \citealt{linFinitenessClausesRaising2011,linMultiplemodalConstructionsMandarin2012,zhangSentencefinalAspectParticles2019}), or are they just heads in the verbal projection of the same clause (mono-clausal, \NNext, \citealt{tsaiTopographyChineseModals2015,yipModalMovementLicensed2022,erlewineLowSentencefinalParticles2017})?
Below, both \Next and \NNext are intended to mean `Lisi can come.'
\begin{paracol}{2}
\ex. \begingl
% \glpreamble bi-clausal //
\gla Lisi \nogloss{\I{[}VP {} } @  \nogloss{\I{[}V {} } @ keyi @ \nogloss{]} \nogloss{\I{[}TP/CP {} } @ \nogloss{\gap{}} lai @ \nogloss{]]} //
  \glb L. can come //
  % \glft `Lisi can come.' //
\endgl

\switchcolumn
\ex.  \begingl
% \glpreamble mono-clausal //
\gla Lisi \nogloss{\I{[}F/ModP \I{[}F/Mod {} } @ keyi @ \nogloss{]} \nogloss{\I{[}\emph{v}P {} } @ \nogloss{\gap{}} lai @ \nogloss{]]} //
  \glb L. can come //
  % \glft `Lisi can come.' //
\endgl

\end{paracol}

% \paragraph{Implication: CP-embedding}
Following \citet{chappellVariationGrammaticalizationComplementizers2008,huangControlComplementsMandarin2018a}, who suggest that \emph{shuo} is a complementizer, we observe that deontic modals can co-occur with it; this is strong evidence that deontic modals are clause(CP)-embedding:
% The claim that root modals embed CPs might be surprising. 
% However, \emph{shuo}, an element that has been claimed by some \citep{chappellVariationGrammaticalizationComplementizers2008,huang2003doubts,huangControlComplementsMandarin2018a} to be potentially a complementizer.
% It is possible with root modals:
\columnratio{.55}
\begin{paracol}{2}
\ex. \begingl
  \gla Lisi keyi \textbf{shuo} xian shi-le zai zuo jueding. //
  \glb L. can \textsc{comp} first try-\textsc{pfv} then make decision //
  % \glft `Lisi can first try it and then make a decision.' //
\endgl

  \switchcolumn
   `Lisi can first try it and then make a decision.' 
\end{paracol}
\columnratio{.50}

This abstract provides three diagnostic tests based on focus-sensitive operators (below, \opfs{}) \emph{dou} and \emph{ye} that further show that deontic modals are clause(CP)-embedding even in the absence of \emph{shuo}.
% The tests in this abstract will further show that even in the absence of \emph{shuo}, the complements of root modals still behave like embedded clauses, even CPs.
This conclusion speaks to the correctness of \citet{wurmbrandImplicationalUniversalComplementation2020} in proposing that the Implicational Complementation Hierarchy refer to minimal structures: while the semantically defined clause-types (\emph{Event}, \emph{Situation}, \emph{Proposition}) have increasing minimum sizes (\emph{v}P, TP, CP), these are not upper bounds; the \emph{Event}-type clauses commonly assumed for such deontic modals can still be CPs.
% This conclusion speaks to the correctness of \citet{wurmbrandImplicationalUniversalComplementation2020} in stating the generalization on clause-types in an implicational way: while the semantically defined clause-types (\emph{Event}, \emph{Situation}, \emph{Proposition}) have increasing minimum sizes (\emph{v}P, TP, CP), there are no upper bounds; the \emph{Event}-type clauses commonly assumed for such deontic modals can still be CPs.
% That root modals embed CPs make the sentences hyperraising, and a \citet{leeHyperraisingEvidentialityPhase2024}-style phase deactivation analysis has been applied to hyperraising in Cantonese, though not with modals, but attitude verbs.
% The CP-embedding nature of root modals, usually thought to take \emph{Event}-type 

% This handout provides a number of arguments in favor of the bi-clausal (clause-embedding) analysis for root modals (deontic, circumstantial, teleological, etc.).
% Whenever the word \emph{modal} is used below, it refers to a root modal.

\paragraph{Test 1: Locality of association with focus}
% \label{sec:argument_from_locality_of_association_with_focus_sensitive_operators}
Test 1 concerns locality conditions of association with \opfs{}; it shows that modals pattern with CP-embedding verbs.
% Specifically, modals should embed CPs.
% First, we establish the locality conditions of FS association without modals.
\Next means `Lisi\F{} even/also came,' showing \opfs{} \emph{ye} `also' and \emph{dou} `even' allow backwards association.
\begin{paracol}{2}
%  \ex. \begingl
% \gla Lisi\F{} \textbf{ye} lai-le. //
%   \glb L. also come-\textsc{pfv} //
%   \glft `Lisi\F{} also came.' //
% \endgl
%
  % \switchcolumn
 \ex. \begingl
\gla  Lisi\F{} \textbf{dou} \nogloss{/} \textbf{ye} lai-le. //
  \glb  L. even also come-\textsc{pfv} //
  % \glft `Lisi\F{} even/also came.' //
\endgl

\switchcolumn
\ex. \label{ex:foc_assoc} \tikzmarknode{target}{\opfs} [ \ldots{} XP\F{} \ldots{} \tikzmarknode{base}{\gap{}} \ldots{} ]
\begin{tikzpicture}[overlay, remember picture]
  \draw[-Stealth] (base.south) -- ++(0, -.25) -| (target.south);
\end{tikzpicture}

\end{paracol}
Following \citet{lahiriFocusNegativePolarity1998,crnicNonmonotonicityNPILicensing2014,liuVarietiesAlternativesMandarin2017}, I assume that \opfs{} must covertly move to a position c-commanding the associate \cref{ex:foc_assoc}.
\Next shows that \opfs{} can move to very high positions in the clause, at least above TopicP.
\emph{Yu} `fish' is a base-generated topic, since there is no gap further down in the sentence.
% \ex. \label{ex:double_island}\begingl
% \gla \nogloss{(} @ lian @ \nogloss{)} Lisi\(_{1, \mathrm{F}}\), \nogloss{\I{[}RC {} } @ xihuan \nogloss{\gap{1}} de @ \nogloss{]} ren \textbf{dou} hen duo. //
%   \glb \textsc{foc} L. like \textsc{de} person even very many //
%   \glft `even Lisi\(_{1, \mathrm{F}}\), the people who like \gap{1} are many.' //
% \endgl
%\gla xihuan Lisi de ren hen duo. + Lisi\F{}, \nogloss{\I{[}RC {} } @ xihuan \nogloss{\gap{}} de @ \nogloss{]} ren ye hen duo. //
%  \glb like L. \textsc{de} person very many L. like \textsc{de} person also very many //
%  \glft `The people who like Lisi are many. Lisi\(_{1, \mathrm{F}}\), the people who like \gap{1} are also many.' //
%\endgl

% \ex. \label{ex:double_island}\begingl
% \gla \nogloss{(} @ lian @ \nogloss{)} Lisi\(_{1, \mathrm{F}}\), \nogloss{\I{[}RC {} } @ xihuan \nogloss{\gap{1}} de @ \nogloss{]} ren \textbf{dou} zhan-man-le yi-ge jiaoshi. //
%   \glb \textsc{foc} L. like \textsc{de} person even stand-full-\textsc{pfv} 1-\textsc{cl} classroom //
%   \glft `even Lisi\(_{1, \mathrm{F}}\), the people who like \gap{1} filled a classroom.' //
% \endgl
%
\begin{paracol}{2}
 \ex. \label{ex:base_topic} \begingl
\gla yu\(_{\mathrm{F}}\), wo \textbf{dou} zhi chi shushi. //
  \glb  fish \textsc{1sg} even only eat cooked.food //
  \glft `Even for fish\F{}, I only eat cooked food.' //
\endgl
 
  \switchcolumn
 \ex. *\begingl
\gla Lisi\F{} shuo \nogloss{\I{[}CP {} } @ Zhangsan \textbf{dou} lai-le @ \nogloss{]} //
  \glb L. say Z. even come-\textsc{pfv} //
  \glft Intended: `Lisi\F{} even said Zhangsan came.' //
\endgl

\end{paracol}


% This is because the gap that is co-indexed with it is embedded in a relative clause in the subject position, i.e., it is doubly island-restricted, a complex-NP island inside a subject island.
% \emph{Lisi} cannot have moved from the gap to its surface position.
% The gap, therefore, is also not created via movement, but an instance of null objects, which are ubiquitous in Mandarin Chinese \citep{liBornEmpty2014}.
Example \Last shows that \opfs{} is unable to associate, and therefore by \cref{ex:foc_assoc}, move, across CP boundaries. 
Then, \LLast and \Last jointly show that \opfs{} movement is subject to \Next:
% \ex. *\begingl
%   \gla Lisi shefa dou lai-le. //
%   \glb L. manage.to even come-\textsc{pfv} //
%   \glft Intended: `Lisi\F{} even managed to come.' //
% \endgl
\ex. \opfs{} movement is basically unrestricted by height in the smallest containing CP, but cannot escape it. \label{ex:loc_cons}

% Thus, the smallest containing CP seems to be the upper bound for \opfs{} movement, as they can already move above TopicP.

% Then, how do \opfs{} behave when modals are present?

Examples \Next and \NNext, meaning `Lisi\F{} can even come,' show that \opfs{} are unable to associate across deontic modals.
From the assumption in \cref{ex:foc_assoc}, this indicates that \opfs{} is unable to move across the modal.
\begin{paracol}{2}
 \ex. \label{ex:fs_root}
\begingl
  \gla Lisi\F{} \textbf{dou} \textbf{keyi} lai. //
  \glb L. even can come. //
  % \glft `Lisi\F{} can also come.' //
\endgl
 
  \switchcolumn
\ex. *\begingl
\gla Lisi\F{} \textbf{keyi} \textbf{dou} lai. //
  \glb L. can even come //
  % \glft Intended: `Lisi\F{} can also come.' //
\endgl

\end{paracol}
% On the other hand, epistemic modals are not barriers to \opfs{} movement:
% \ex. \label{ex:fs_epistemic}
% \a. \begingl
%   \gla Lisi\F{} \textbf{ye} \textbf{yinggai} lai-le. //
%   \glb L. also should come. //
%   \glft `Lisi\F{} should also have come.' //
% \endgl
% \b. \begingl
% \gla Lisi\F{} \textbf{should} \textbf{ye} lai-le. //
%   \glb L. can also come //
%   \glft Intended: `Lisi\F{} can also come.' //
% \endgl




% \Last[a] just shows us that if the \opfs{} and its associate are on the same side of a modal, then association is possible.

% Examples similar to \cref{ex:double_island} also becomes unacceptable when a modal is inserted somewhere between the topicalized associate and the \opfs{}:
% \ex. \label{ex:double_island_modal} 
% \a. \begingl
% \gla \nogloss{(} @ lian @ \nogloss{)} Lisi\(_{1, \mathrm{F}}\), \nogloss{\I{[}RC {} } @ xihuan \nogloss{\gap{1}} de @ \nogloss{]} ren \textbf{dou} \textbf{keyi} zhan-man yi-ge jiaoshi. //
%   \glb \textsc{foc} L. like \textsc{de} person even can stand-full 1-\textsc{cl} classroom //
%   \glft `even Lisi\(_{1, \mathrm{F}}\), the people who like \gap{1} can fill a classroom.' //
% \endgl
% \b. \begingl
% \gla \nogloss{\ljudge{*}} @ \nogloss{(} @ lian @ \nogloss{)} Lisi\(_{1, \mathrm{F}}\), \nogloss{\I{[}RC {} } @ xihuan \nogloss{\gap{1}} de @ \nogloss{]} ren \textbf{keyi} \textbf{dou} zhan-man yi-ge jiaoshi. //
%   \glb \textsc{foc} L. like \textsc{de} person can even stand-full 1-\textsc{cl} classroom //
%   \glft `even Lisi\(_{1, \mathrm{F}}\), the people who like \gap{1} can fill a classroom.' //
% \endgl
%
% \ex. \label{ex:double_island_modal}
%\a. \begingl
%\gla zhichi Lisi de ren keyi zai zheli tou piao, //
%  \glb support L. \textsc{de} person can at here cast vote //
%  \glft `People who support Lisi can vote here, ' //
%\endgl
% \a. \begingl
% \gla \nogloss{\ldots{}} Lisi\F{}, \nogloss{\I{[}RC {} } @ zhichi \nogloss{\gap{}} de @ \nogloss{]} ren \textbf{ye} \textbf{keyi} zai zheli tou piao. //
%   \glb L. support \textsc{de} person also can at here cast vote //
%   \glft `Lisi\(_{1, \mathrm{F}}\), people who support \gap{1} can also vote here.' //
% \endgl
% \b. \begingl
% \gla \nogloss{\ljudge{*}} @ \nogloss{\ldots{}} Lisi\F{}, \nogloss{\I{[}RC {} } @ zhichi \nogloss{\gap{}} de @ \nogloss{]} ren \textbf{keyi} \textbf{ye} zai zheli tou piao. //
%   \glb L. support \textsc{de} person also can at here cast vote //
%   \glft Intended: `Lisi\(_{1, \mathrm{F}}\), people who support \gap{1} can also vote here.' //
% \endgl

Thus, deontic modals pattern with CP-embedding verbs in forbidding \opfs{} movement out of them.
Without adding ad hoc modal-specific constraints, the simplest explanation is that modals are also CP-embedding verbs.
% Because \cref{ex:base_topic} shows that \opfs{} movement is not confined to within TP, the complement clauses of modals must be CPs.
This way, \cref{ex:loc_cons} can directly apply to derive the contrast in \LLast and \Last.

% Modals pattern with clause-embedding verbs in forbidding the movement of \opfs{} across them.
%
% If we do not want to introduce new ad hoc constraints on the movement of \opfs{}, the only option is to assume that modals also embed clauses.


\paragraph{Test 2: Scope of \opfs{}}

A separate but related test concerns the scopal possibility of embedded \opfs{} w.r.t. deontic modals.
Consider \Next, which is correctly predicted to be grammatical by Test 1 (both the \opfs{} and its associate are on the same side of the modal; \cref{ex:loc_cons} is not violated).
\columnratio{.55}
\begin{paracol}{2}
 \ex. \begingl
\gla Lisi keyi lunwen\F{} \textbf{ye} mingtian xie. //
  \glb L. can paper also tomorrow write //
  \glft `Lisi is allowed to also write the paper\F{} tomorrow.' //
\endgl

  \switchcolumn
 \ex. \begingl
  \gla Lisi xiwang Zhangsan\F{} \textbf{ye} lai. //
  \glb L. hope Z. also come //
  \glft `Lisi hopes that Zhangsan\F{} will also come.' //
\endgl

\end{paracol}
\columnratio{.5}
Example \LLast necessarily presupposes the epistemic statement that Lisi is \emph{expected} to write something else tomorrow.
The presupposition is derived if we assume that 1) \emph{ye} is interpreted below \emph{keyi}, and that 2) presuppositions project universally and epistemically from under deontic modals in the manner described in \citet{heimPresuppositionProjectionSemantics1992}.
The reading where \emph{ye} outscopes \emph{keyi}, merely presupposing that Lisi is \emph{allowed} to write something else tomorrow, is unavailable.
\LLast is infelicitous where only the deontic presupposition is met.
Again, modalized sentences pattern with clause-embedding verbs \Last.
% \ex. \begingl
% \gla Lisi juede Zhangsan\F{} \textbf{ye} lai-le. //
%   \glb L. think Z. also come-\textsc{pfv} //
%   \glft `Lisi thinks Zhangsan\F{} also came.' //
% \endgl
%
\emph{Ye} must be interpreted under \emph{xiwang} in \Last.
It is presupposed that Lisi \emph{believes} someone other than Zhangsan will come; it is infelicitous where Lisi only \emph{hopes} someone other than Zhangsan will come.

% When \opfs{} do appear in the syntactic scope of modals, they get interpreted in the semantic scope of modals as well, because they are unable to escape the embedded CP that is the complement of the modal.
% And because they must c-command their associate, they can then only associate with elements also in the syntactic scope of the modals. 
% We have already seen examples where the \opfs{} in the scope of modals fail to associate with elements outside it.
% Now we look at an example where there is a clause-mate associate for an embedded \opfs{}.

% The additivity is clearly interpreted inside \emph{keyi}.
% \Last contrast with a sentence where both the \opfs{} and the associate are outside the scope of the modal:
% \ex. \begingl
% \gla Lisi \nogloss{[} @ zhe-fen lunwen @ \nogloss{]\F{}} \textbf{ye} keyi mingtian xie. //
%   \glb L. this-\textsc{cl} paper also can tomorrow write //
%   \glft `Lisi can also do this paper\F{} tomorrow.' //
% \endgl
%
% \Last presupposes the deontic statement that Lisi is \emph{allowed} to do at least one other paper tomorrow.
%
% This is derived by calculating the semantic contribution of \emph{ye} outside the scope of the modal.

% In a context where Lisi is allowed to do another paper tomorrow, but doesn't plan to do so, and so no one in the context expects him to actually do so, \LLast is infelicitous, while \Last is felicitous.


\paragraph{Test 3: Distribution of \opfs{}}


% The second argument involving the ``resetting'' of ordering constraints between \opfs{} and fronted objects by modals.

Test 3 concerns the ability of deontic modals to allow \opfs{} to precede items that they otherwise cannot.
Observe that \emph{ye} `also' and \emph{dou} `even,'  which are always pre-verbal, cannot precede fronted objects, which I assume, following \citet{chenObscuredUniversalityMandarin2023}, to be TP-internal topics, situated in what I call Spec, \topobj.
Both \Next and \NNext are intended to mean `Lisi\F{} also wrote the paper.'
% \ex. Pre-verbal adverbials
% \a. \begingl
% \gla Lisi\F{} zuotian ye lai-le. //
%   \glb L. yesterday also come-\textsc{pfv} //
% \glft `Lisi\F{} also came yesterday.' //
% \endgl
% \b. *\begingl
%   \gla  Lisi\F{} ye zuotian lai-le. //
%   \glb L. also yesterday come-\textsc{pfv} //
%   \glft Intended: `Lisi\F{} also came yesterday.' //
% \endgl
%
\begin{paracol}{2}
 \ex. \begingl
\gla Lisi\F{} \textbf{lunwen}\(_1\) \textbf{ye} xie-le \nogloss{\gap{1}}. //
  \glb L. paper also write-\textsc{pfv} //
  % \glft `Lisi\F{} also wrote the paper.' //
\endgl
 
  \switchcolumn
 \ex. * \begingl
\gla Lisi\F{} \textbf{ye} \textbf{lunwen}\(_1\) xie-le \nogloss{\gap{1}}. //
  \glb L. also paper write-\textsc{pfv} //
  % \glft Intended: `Lisi\F{}  did his paper' //
\endgl

\end{paracol}


Then, Spec, \topobj{} must be in a higher position than the base (overt) position of \opfs{}.
\begin{paracol}{2}

  This is represented in the sequence in \Next:
  \switchcolumn
  \ex. \label{ex:order_top_fs} \topobj{} \(\prec\) \opfs{} \(\prec\) V

\end{paracol}

When modals are involved, however, \opfs{} can attach to the immediate left of the modals, even above positions originally impossible for attachment.
  Ignoring the modal, \Next has the sequence in \NNext:
% \ex.
% \begingl
% \glpreamble Pre-verbal adverbials //
% \gla Lisi\F{} ye keyi mingtian lai. //
%   \glb L. also can tomorrow come //
% \glft `Lisi\F{} can also come tomorrow.' //
% \endgl
%
\columnratio{.6}
\begin{paracol}{2}
\ex.
\begingl
% \glpreamble Fronted objects //
\gla Lisi\F{} \textbf{ye} \textbf{keyi} \textbf{lunwen}\(_1\) mingtian xie \nogloss{\gap{1}}. //
  \glb L. also can paper tomorrow write //
  \glft `Lisi\F{} is also allowed to write the paper tomorrow.' //
\endgl

  \switchcolumn
  \ex. \opfs{} \(\prec\) \topobj{} \(\prec\) V  \label{ex:paradox}

\end{paracol}

\columnratio{.5}
The sequence in \Last is in conflict with \cref{ex:order_top_fs}.
This can be easily explained if modals embed complement clauses at least the size of \topobj{}.
For simplicity and in accordance with Test 1, I assume they embed CPs.
\begin{paracol}{2}

  The sequence in \Last can then be rewritten as \Next:
  \switchcolumn
  \ex. \I{[}CP \opfs{} \(\prec\) \vmod{} \(\prec\) \I{[}CP \topobj{} \(\prec\) V  ]]

\end{paracol}

In each CP in \Last, \cref{ex:order_top_fs} is not violated.
An immediate prediction is that as long as the \opfs{} and \topobj{} are not separated by the clausal boundary that the modal induces, they must still obey \cref{ex:order_top_fs}.
This prediction is borne out.
First, if the object is fronted across the modal, it forms a sequence with \opfs{} that obeys \cref{ex:order_top_fs}: 
% Expectedly, matrix fronted objected must still precede matrix \opfs{}:
% \ex. Pre-verbal adverbials
% \a. \begingl
% \gla Lisi\F{} mingtian ye keyi lai. //
%   \glb L. tomorrow also can come //
% \glft `Lisi\F{} can also come tomorrow.' //
% \endgl
% \b. *\begingl
% \gla Lisi\F{} ye mingtian keyi lai. //
%   \glb L. also tomorrow can come //
% \glft `Lisi\F{} can also come tomorrow.' //
% \endgl
%
\ex. \begingl
\gla Lisi\F{} \nogloss{\{} @  \textbf{lunwen}\(_1\) \textbf{ye} \nogloss{/} \nogloss{*} @ \textbf{ye} \textbf{lunwen}\(_1\)  @ \nogloss{\}} keyi mingtian xie \nogloss{\gap{1}}. //
  \glb L. paper also also paper can tomorrow write //
  \glft `Lisi\F{} is also allowed write the paper tomorrow.' //
\endgl


% \begin{paracol}{2}
%  \ex. \begingl
% \gla Lisi\F{} \textbf{lunwen}\(_1\) \textbf{ye} keyi mingtian xie \nogloss{\gap{1}}. //
%   \glb L. paper also can tomorrow write //
%   \glft `Lisi\F{} can also write his paper tomorrow.' //
% \endgl
%
%   \switchcolumn
% \ex. \begingl
% \gla \nogloss{\ljudge{*}} @ Lisi\F{} \textbf{ye} \textbf{lunwen}\(_1\) keyi mingtian xie \nogloss{\gap{1}}. //
%   \glb L. also paper can tomorrow write //
%   \glft `Lisi\F{} can also write his paper tomorrow.' //
% \endgl
%
% \end{paracol}
\begin{paracol}{2}
  
  Example \Last illustrates the sequence \Next:
  \switchcolumn
  \ex. \I{[}CP \topobj{} \(\prec\) \opfs{} \(\prec\) \vmod{} \(\prec\) \I{[}CP V  ]] \label{ex:order_matrix}

\end{paracol}

Second, if \topobj{} and \opfs{} are both in the scope of the modal, \cref{ex:order_top_fs} is also followed:
\ex.
\begingl
\gla Lisi keyi \nogloss{\{} @  \textbf{lunwen}\(_{1, \mathrm{F}}\) \textbf{ye} @ \nogloss{/} \nogloss{*} @ \textbf{ye} \textbf{lunwen}\(_{1, \mathrm{F}}\) @ \nogloss{\}} mingtian xie \nogloss{\gap{1}}. //
  \glb L. can  paper also also  paper tomorrow write //
  \glft `Lisi is allowed to also write the paper\F{} tomorrow.' //
\endgl


% \begin{paracol}{2}
%   
% \ex.
% \begingl
% \gla Lisi keyi \nogloss{[} @  zhe-fen lunwen @ \nogloss{]\(_{1, \mathrm{F}}\)} ye mingtian xie \nogloss{\gap{1}}. //
%   \glb L. can this-\textsc{cl} paper also tomorrow write //
%   \glft `Lisi can also do this paper\F{} tomorrow.' //
% \endgl
%
%   \switchcolumn
% \ex. \begingl
% \gla \nogloss{\ljudge{*}} @ Lisi keyi ye \nogloss{[} @  zhe-fen lunwen @ \nogloss{]\(_{1, \mathrm{F}}\)} mingtian xie \nogloss{\gap{1}}. //
%   \glb L. can also this-\textsc{cl} paper tomorrow write //
%   \glft Intended: `Lisi can also do this paper\F{} tomorrow.' //
% \endgl
%
% \end{paracol}

\begin{paracol}{2}

  Example \Last illustrates the sequence \Next:
  \switchcolumn
  \ex. \I{[}CP \vmod{} \(\prec\) \I{[}CP \topobj{} \(\prec\) \opfs{} \(\prec\) V  ]] \label{ex:order_embedded}

\end{paracol}

Clearly, both \cref{ex:order_matrix} and \cref{ex:order_embedded} follow \cref{ex:order_top_fs}.
This means that modals essentially `reset' the clausal projections and therefore the sequence that \opfs{} and \topobj{} follow.

% Thus, we can summarize the ordering constraints with modals into the following:
% \ex. \topobj{} \(\prec\) \opfs{} \(\prec\) modal \(\prec\) \topobj{}  \(\prec\) \opfs{}
%
% But crucially, without the modal, only the relationship \Next holds.
% \ex.  \topobj{} \(\prec\) \opfs{}
%
% This can be easily explained by the bi-clausal analysis of modals.
%
% The ordering constraint \cref{ex:order_top_fs} applies in every clause.
%
% But modal constructions involve two clauses, one embedded, and one matrix.
%
% In each of them, \cref{ex:order_top_fs} must hold.
%
% But if we choose to have \opfs{} in the matrix clause, and \topobj{} in the embedded clause, we can derive the seemingly paradoxical \cref{ex:paradox}.
%
% This means that modals essentially `reset' the clausal projections and therefore the ordering constraints with \opfs{}.






% \paragraph{Implication: CP-embedding}


% The claim that root modals embed CPs might be surprising. 
% However, \emph{shuo}, an element that has been claimed by some \citep{chappellVariationGrammaticalizationComplementizers2008,huang2003doubts,huangControlComplementsMandarin2018a} to be potentially a complementizer.
% It is possible with root modals:
% \ex. ?\begingl
%   \gla Lisi yinggai \textbf{shuo} xian shi-yi-shi zai zuo jueding. //
%   \glb L. should \textsc{comp} first try then make decision //
%   \glft `Lisi should first try it and then make a decision.' //
% \endgl

% That root modals embed CPs make the sentences hyperraising, and a \citet{leeHyperraisingEvidentialityPhase2024}-style phase deactivation analysis has been applied to hyperraising in Cantonese, though not with modals, but attitude verbs.
% The CP-embedding nature of root modals, usually thought to take \emph{Event}-type 

% \paragraph{Implication: Modal movement}
%
% \citet{yipModalMovementLicensed2022,laiMovingHeadsSpecifiers2024} propose that Mandarin modal-initial sentences are derived through modal movement across the subject \Next.
% \begin{paracol}{2}
%   \ex. \tikzmarknode{mod}{Mod} \ldots{} \I{[}TP Subj \tikzmarknode{base}{\gap{}} \I{[}\emph{v}P \ldots{} ]
% \begin{tikzpicture}[overlay, remember picture]
%   \draw[Stealth-] ([xshift=1.5mm, yshift=-.8mm]mod.north) -- ++(0, .25) -| (base.north);
% \end{tikzpicture}
%
%   \switchcolumn
% \ex. V\(_{\mathrm{Mod}}\) \I{[}CP \ldots{} Subj \I{[}T\('\) \ldots{} ]]
%
% \end{paracol}
%
% But if modals are raising verbs, then modal-initial sentences are much more likely to be just constructions where the subject does not raise \Last, though a full analysis of the relevant data remains to be developed.
% % This is then the analysis of \citet{linFinitenessClausesRaising2011} for such sentences, but he
%
\paragraph{Extension: Epistemic modals}
\label{par:extension_epistemic_modals}

The abstract has dealt primarily with deontic modals.
The relevant tests can be extended to epistemic modals.
However, the tests give contradictory results.
\begin{paracol}{2}
  \ex. 
  \begingl
  \gla Lisi\F{} \textbf{ye} \textbf{keneng} lai-le. //
  \glb L. also may come-\textsc{pfv} //
  \glft `Lisi\F{} may also have come.' \hfill \(\text{also} > \Diamond\) //
  \endgl

  \switchcolumn
  \ex. \begingl
  \gla Lisi\F{} \textbf{keneng} \textbf{ye} lai-le. //
  \glb L. may also come-\textsc{pfv} //
  \glft `Lisi\F{} may have also come.'\hfill \(\Diamond > \text{also}\) //
  \endgl

\end{paracol}


Test 1 shows that epistemic modals are not barriers for focus association, as \Last is grammatical.
Test 2 shows that the scope of \opfs{} w.r.t. epistemic modals is still determined by their surface order, indicating that \opfs{} cannot move across epistemic modals.
Test 3 shows that epistemic modals do allow \opfs{} to precede fronted objects, indicating a second clausal projection.
The easiest way to explain away the odd one out, Test 1, is to assume that 1) epistemic modals are like deontic modals in being CP-embedding verbs, that 2) epistemic modals are raising verbs, and that 3) epistemic modals alone allow reconstruction of raised subjects into the embedded CP, enabling \opfs{}, which cannot escape the CP, to associate with the reconstructed subject.


% \subsection{Pre-subject}
% \label{sub:pre_subject}
%
% If modals are Vs embedding clauses, then they are either raising or control verbs, due to the possibility of a matrix subject co-indexed with the embedded null subject position.
%
% That almost every modal allows for the subject to stay in the embedded position in an A-not-A configuration and that fact that these modals do not seem to assign any meaningful \(\theta\)-roles mean that the control analysis is a non-starter.



% \subsection{Deriving the syntactic behavior of \emph{dou} across the uses}
% \label{sub:deriving_the_syntactic_behavior_of_dou_across_the_uses}
%
% Recall that I provided the DLC in \cref{itm:dlc}, repeated in \Next:
% \ex. \emph{\emph{Dou} Locality Constraint} (DLC)\\
% \emph{Dou} and its associate cannot be linearly intervened by a scope-taking element, such as a modal or negative marker.
%
% This is used as evidence against Liu's analysis of the distributive use of \emph{dou} where \emph{dou} associates with the plural.
% I will show that my re-analysis does indeed make it possible for the distributive use to still conform to this constraint.
% Besides, there should be ways to state the DLC in more explicitly structural and derivational terms, perhaps even dissolving its stipulative nature through appropriate structural assumptions.
%
% The ingredients to the structural analysis of \emph{dou} are as follows.
% \ex. \label{itm:dou_struct}
% \a. \emph{Dou} moves from its overt position in the lower verbal domain to a high CP-adjoined. \label{itm:vp}
% \b. The movement of \emph{dou} cannot cross CP boundaries. \label{itm:cp}
% \b. \emph{Dou} and its the associate must be clausemates at surface. \label{itm:mate}
% \b. Scope-taking elements such as modals and negation are verbal and take CPs as complements. \label{itm:v}
% \b. The movement of \emph{dou} must cross its associate except in the scalar use. \label{itm:cross}
%
% I will then motivate these structural specifications one by one.
%
% \Last[a] is based on the fact that \emph{dou} can associate with elements base-generated high in the clause.
% For example, \emph{dou} can associate with a base-generated topic, which corresponds with a gap that is inside an island, for example, a relative class in the subject position:
% \ex. \gll
% \textbf{dou} \I[TopP zhe-ge ban de mei-ge\(_{D_F}\) ren \I[Top\('\) \I[RC xihuan \gap{1} de ren ] \textbf{\sout{dou}} hen duo ]].\\
% \textsc{dou} {} this-\textsc{cl} class \textsc{de} every-\textsc{cl} person {} {} like {} \textsc{de} person {} \textsc{dou} very many\\
% \glt `Everyone \(x\) in this class is such that there are many people who like \(x\).' \label{itm:no-movement}
%
% Since this gap cannot be derived through movement as it is embedded inside two islands, the topic that seems dislocated is base-generated and the gap should be a case of object-drop, ubiquitous in Mandarin Chinese, anaphoric to the topic.
% Therefore, \emph{dou} is at least able to move as high as just above TopP.
%
% \cref{itm:cp} and \cref{itm:mate} are expected of the \emph{even}-like elements, with evidence from both Mandarin Chinese and English.
% \ex. *\gll
% mei-ge\(_{D_F}\) ren shuo \I[CP Lisi dou xie-le lunwen ]. \\
% every-\textsc{cl} person say {} L. \textsc{dou} wrote homework \\
% \glt Intended: `Everyone said that Lisi did his homework.'
%
% If \emph{dou} were allowed to move into the matrix CP, or if \emph{dou} were allowed to associate with an element outside the smallest containing CP, then nothing prevents \emph{dou} from associating with \(D_F\) on \emph{mei} in the matrix, failing to predict the ungrammaticality of \Last.
% The same kind of example can be readily replicated in English:
% \ex. *John\(_F\) said Bill \textbf{even} did his homework.
%
% \cref{itm:cross} is needed simply to encode the fact that \emph{dou} obligatorily associates to the left except in the scalar use.
% When the focus associate is in the lower verbal domain at surface, then a covert counterpart of \emph{dou} is used.
% We will nevertheless focus on cases where overt \emph{dou} is involved.
%
%
% \cref{itm:v} is a natural consequence of \cref{itm:cp} and \cref{itm:mate} and the fact that \emph{even}-type elements are cross-linguistically able to move out of the scope of negation and modals.
% Take English for example:
% \ex. \textbf{even} John didn't \textbf{\sout{even}} read any book.
%
% \emph{Even} needs to move out of the scope of negation for its presupposition that the prejacent is the strongest to be true.
% In \Next, \emph{even} also seems to associate with an element across the modal \emph{can}, indicating that it is able to move across it.
% \ex. John\(_F\) can \textbf{even} swim.
%
%
% I have shown that \emph{dou} should be able to move as high as the edge of a CP in \cref{itm:cp}.
% If there is no other principle that prohibits the movement of \emph{dou} across modals and negation, it must be \cref{itm:cp} that is at work.
% Then, it must be the case that negation and modals in Mandarin Chinese take CPs as complements; then, \emph{dou} will not be able to move across them.
% This also means that negation and modals are verbal and are located in the lower verbal domain; the subject of the complement clause is raised across them into the matrix clause.
% This is evidenced by the very fact that \emph{dou} can appear above them in the same clause:
% \ex. \gll
% mei-ge\(_{D_F}\) ren\(_1\) \textbf{dou} \I[VP keyi \I[CP \gap{1} lai ]]. \\
% every-\textsc{cl} person \textsc{dou} {} can {} {} come \\
% \glt `They all can come.'
%
% These observed and deduced syntactic constraints on \emph{dou} and its associate depart from those reported in \citet{erlewineMovementOutFocus2014a} for English \emph{even}.
% First, we have \cref{itm:no-movement} which is evidence for \emph{dou} associating with an element that is never in the scope of the surface position of \emph{dou}.
% Having a copy inside the surface scope of \emph{dou} is not a \emph{necessary} condition for the association.
% Second, we have already seen that the Chinese counterparts to \Last and \LLast with negation and the modal intervening between the subject associate and \emph{dou} are ungrammatical.
% Having a copy inside the surface scope of \emph{dou} is not a \emph{sufficient} condition for the association.
% Therefore, we can conclude that the Chinese data motivates a syntactic theory of association with focus that involves either movement of \emph{even}/\emph{dou} to a high position in the clause (at least above TopP), becoming a scope-taker with the enlarged scope, or is a reflex of such a scope-taker that is base-generated in a high position.
% For concreteness and familiarity, I have assumed that \emph{dou} itself moves.
% I leave to future research the reconciliation between the Chinese data and the theory of focus association in \citet{erlewineMovementOutFocus2014a}.
% % \Last[b] is a natural consequence of \Last[a].
% % \Last[c] and \Last[e] will conspire to derive the DLC.
% % \Last[d] derives \emph{dou}'s leftward association except in the scalar use, where the association can be either leftward or rightward.
% Then, we see that we can derive the DLC from \crefrange{itm:vp}{itm:cross}: \emph{dou} cannot escape its own minimal CP, and cannot associate across CP boundaries; modals and negation take CPs as complements, so whenever a modal or negation intervenes between \emph{dou}'s supposed associate and \emph{dou}, they will be separated by a CP boundary and association cannot be established between them.
%
% Thus, I can provide more detailed structures for the various uses of \emph{dou}, deriving the relevant grammaticality judgements.
% I will start with the distributive use.
% Because \emph{dou} now associates with the domain variable \(D\) on \(\Delta\) and the association is always leftward, we can put \(\Delta\) in an appropriate clausemate position:
% \ex. \gll
% dou tamen\(_1\) \(\Delta_{D_F}\) \sout{dou} \I[VP keyi \I[CP \gap{1} \I[VP lai ]]]. \\
% \textsc{dou} they {} {} {} can {} {} {} come \\
% \glt `They are all allowed to come.'\\
% \(\forall > \Diamond\)
%
% \ex. \gll
% tamen\(_1\) \I[VP keyi \I[CP dou \gap{1} \(\Delta_{D_F}\) \sout{dou} \I[VP lai ]]]. \\
% they {} can {} \textsc{dou} {} {} {}  {} come \\
% \glt `They are allowed to all come.'\\
% \(\Diamond > \forall\)
%
% When \emph{dou} is to the left of a modal or negation, it must be in a higher clause; by \cref{itm:mate}, \(\Delta\) will also be in the higher clause, making \(\Delta\) scope above said modal or negation.
% When \emph{dou} is to the right of a modal or negation, it must be in the complement clause of said modal or negation; by \cref{itm:mate}, \(\Delta\) will also be in the complement clause, making \(\Delta\) scope below said modal or negation.
% Thus, the two scope relations in \LLast and \Last are derived.
%
%
% % Modals and negation(!) are verbal and take clausal complements.
% %
% % Some modals are obligatorily raising, others optionally.
% % Negation (\emph{bu} and \emph{mei}, not \emph{bushi}) is obligatory raising, can never precede the subject.
% % Also necessitated by the assumption that \emph{dou} can only occupy a position on the left edge of VP or \emph{v}P
%
% % \emph{Dou} and its associate must be clausemates overtly; traces of the associates don't count for this purpose.
% % Movement of \emph{dou} cannot across any CP boundaries.
% % Movement of \emph{dou} must across its associate in all but the scalar use; this derives the leftward association requirement.
% % This is perhaps related to (Generalized) Scope Economy.
% % Distributive use: even though the  plural can be separated from \emph{dou} by negation or modals, the true associate, \(\Delta\), can always be situated in the same clause as \emph{dou}.
% % \ex. \gll
% % dou tamen\(_1\) \(\Delta_F\) \sout{dou} \I[VP keyi \I[CP \gap{1} \I[VP lai ]]]. \\
% % {} they {} \textsc{dou} {} can {} {} {} come \\
% % \glt `They all can come.'\\
% % \(\forall > \Diamond\)
% %
% % \ex. \gll
% % tamen\(_1\) \I[VP keyi \I[CP dou \gap{1} \(\Delta_F\) \sout{dou} \I[VP lai ]]]. \\
% % they {} can {} {} {} {} \textsc{dou} {} come \\
% % \glt `They are allowed to all come.'\\
% % \(\Diamond > \forall\)
%
% % \ex. \gll
% % tamen\(_1\) dou \I[VP mei \I[CP \gap{1} \I[VP lai ]]]. \\
% % they \textsc{dou} {} \textsc{neg} {} {} {} come \\
% % \glt `They all didn't come.'\\
% % \(\forall > \neg\)
% %
% % \ex. \gll
% % tamen\(_1\) \I[VP mei \I[CP \gap{1} dou \I[VP lai ]]]. \\
% % they {} \textsc{neg} {} {} \textsc{dou} {} come \\
% % \glt `They haven't all come.'\\
% % \(\neg > \forall\)
% %
% In the other uses, \emph{dou}'s associate can be an overt clausemate of \emph{dou} when no modals or negation intervenes; however,
% \emph{dou}'s associate cannot be an overt clausemate of \emph{dou} when \emph{dou} and the associate are separated by negation or modals because the associate must have moved out of the clause where \emph{dou} is located due to raising \NNext.
% Thus the sentence is ungrammatical.
% \ex. \gll
% dou renhe ren\(_{1, F}\) \sout{dou} keyi \I[CP \gap{1} lai ]. \\
% {} any person \textsc{dou} can {} {} come \\
% \glt `Any person can come.'
%
% \ex. * \gll
% renhe ren\(_{1, F}\) keyi \I[CP dou \gap{1} \sout{dou} lai ]. \\
% any person can {} {} {} \textsc{dou} come \\
% \glt Intended: `Any person can come.'
%
% % \ex. \gll
% % dou (lian) Lisi\(_{1, F}\) \sout{dou} keyi \I[CP \gap{1} lai]. \\
% % {} \phantom{(}\textsc{lian} L. \textsc{dou} can {} {}  come \\
% % \glt `Even Lisi can come.'
% %
% % \ex. * \gll
% % (lian) Lisi\(_{1, F}\) keyi \I[CP dou \gap{1} \sout{dou} lai]. \\
% % \phantom{(}\textsc{lian} L. can {} {} {} \textsc{dou} come \\
% % \glt Intended: `Even Lisi can come.'
% %
%
% Thus, whether the associate of \emph{dou} is realized as an overt element separate from \emph{dou} becomes the main factor determining the syntactic behavior of \emph{dou} with respect to modals and negation.
% \(\Delta_D\) is covert and the stress is realized on \emph{dou}; therefore, it can be situated in whichever clause \emph{dou} is situated in.
% Thus, its scope is indicated by the position of \emph{dou}.
% Wherever the modal or negation is, \(\Delta_D\) can always be placed right next to \emph{dou}, avoiding any violation of the DLC.
% But the focus associate of the scalar use of \emph{dou}, \emph{mei}\(_D\), and \emph{renhe}\(_D\) are overt and independent from \emph{dou}, so in some configurations there is no way to put them in the same clause as \emph{dou}, making association with \emph{dou} impossible because of the DLC.
%
%
\newpage
This abstract should be considered for the main session. If admitted as a talk/poster, it will be presented in person.
\printbibliography


\end{document}
