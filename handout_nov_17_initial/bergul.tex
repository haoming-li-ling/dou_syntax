\documentclass[11pt]{article}
\usepackage[letterpaper, margin=1in]{geometry}
% \usepackage[skip=\baselineskip, tocskip=0pt]{parskip}
\usepackage{amsmath, amsfonts, amssymb, amsthm, mathtools}
\usepackage{stmaryrd}
\usepackage{fontspec}
\let\latextextsubscript\textsubscript
\AtBeginDocument{
  \let\textsubscript\latextextsubscript
  \newcommand{\sub}[1]{\textsubscript{#1}}
  \newcommand{\gap}[1]{\rule{1em}{0.4pt}\textsubscript{#1}}
}
\let\latextextsuperscript\textsuperscript
\AtBeginDocument{
  \let\textsuperscript\latextextsuperscript
}
\usepackage[largesc]{newpxtext}
\usepackage[vvarbb]{newpxmath}
\usepackage{csquotes}
\usepackage{markdown}

\usepackage{pifont}
% \usepackage{txfonts}
\let\eachwordone=\rmfamily
\let\eachwordtwo=\rmfamily
\let\eachwordthree=\rmfamily
\let\rm=\relax
% \usepackage{linguex, cgloss}
\usepackage[linguex]{expex-glossonly}
\lingset{everygla={\itshape}}
\renewcommand{\firstrefdash}{}
\usepackage{hyperref}
\usepackage{cleveref}
\crefname{ExNo}{}{}
\crefname{SubExNo}{}{}
\renewcommand{\theExNo}{\arabic{ExNo}}
\renewcommand{\theSubExNo}{\theExNo\alph{SubExNo}}
\creflabelformat{SubExNo}{(#2#1#3)}
\creflabelformat{ExNo}{(#2#1#3)}
\crefrangelabelformat{SubExNo}{(#3#1#4--#5\crefstripprefix{#1}{#2}#6)}
\crefrangelabelformat{ExNo}{(#3#1#4)--(#5#2#6)}
\usepackage{enumitem}
\usepackage[normalem]{ulem}
\usepackage{xcolor}
\setlength{\parskip}{1ex}
\setlength{\parindent}{0pt}
\setlist[1]{left=0pt}
% \renewcommand\thesection{\Roman{section}.}
% \renewcommand\thesubsection{\arabic{subsection}.}
% \renewcommand\thesubsubsection{\Alph{subsubsection}}

\usepackage[
  backend=biber,
  natbib=true,
  style=unified,
  maxcitenames=3,
  maxbibnames=99
]{biblatex}
\addbibresource{../Chinese_Modals.bib}
\AtBeginDocument{
  % \setlength{\Extopsep}{0\baselineskip}
  % \setlength{\Exredux}{0\baselineskip}
  \settowidth{\Exlabelwidth}{(00)}
  % \setlength{\Exlabelwidth}{.7\Exlabelwidth}
  % \setlength{\Exlabelsep}{.7\Exlabelsep}
  % \setlength{\SubExleftmargin}{\SubExleftmargin}
  % \setlength{\SubSubExleftmargin}{\SubSubExleftmargin}
}

\newcommand{\II}{\I{[}}
\newcommand{\A}{\(\overline{\text{A}}\)}
\usepackage{tikz}
\usepackage[linguistics]{forest}
\usetikzlibrary{positioning}
\usetikzlibrary{arrows.meta}
\usetikzlibrary{tikzmark}
\usepackage{tikz-cd}
\tikzcdset{
  arrow style=math font,
  % diagrams={>={Straight Barb[scale=0.8]}}
}

\tikzset{every label/.style={font=\footnotesize}}

\forestset{
  default preamble={
    for tree={
      inner sep=0pt,
      % draw,
    }
  },
  great empty nodes/.style={
    for tree={
      calign=fixed edge angles,
      calign primary angle=-60,
      calign secondary angle=60, 
    l=7mm},
    delay={
      where content={}{
        shape=coordinate,
    for current and siblings={anchor=north}}{}}
  },
  downroof/.style={
    for children={
      if n=1{
        edge path'={
          (.parent first) -- (!u.parent anchor) -- (!ul.parent last) -- cycle
        }
      }{no edge}
    }
  }
}
\usepackage{tabularx}
\usepackage{booktabs}
\renewcommand\tabularxcolumn[1]{m{#1}}% for vertical centering text in X column

\AtBeginDocument{% to do this after unicode-math has done its work
  \renewcommand{\setminus}{\mathbin{\backslash}}%
}
\title{Arguments for a bi-clausal analysis of Mandarin root modals}
\author{Haoming Li}

\DeclareMathOperator{\atom}{\textsc{Atom}}
\DeclareMathOperator{\alt}{\textsc{Alt}}
\DeclareMathOperator{\xh}{\textsc{Exh}}
\newcommand{\exh}{\ensuremath{\xh}}
\newcommand{\Exh}{\ensuremath{\mathcal{E}\mathit{xh}}}
\newcommand{\Pex}{\ensuremath{\mathcal{P}\mathit{ex}}}
\newcommand{\F}{\ensuremath{_{\mathrm{F}}}}
\DeclareMathOperator{\prt}{Part}
\DeclareMathOperator{\dom}{dom}

\begin{document}
\section{Contrastive topic}
\label{sec:contrastive_topic}

\ex. \gll 
Yuehan yao qu paidui. Mali ne, ta yao hui jia. \\
John will go party Mary \textsc{ct} \textsc{3sg} will return home \\
\glt `John will go to the party. As for Mary, she will go home.'

See Constant (2014)

\section{O-marked conditional and FLV}
\label{sec:o_marked_conditional}

No distinction between O-marked and FLV. Replacing \emph{ruguo} with \emph{yaoshi} could be more FLV-like (\emph{yaoshi} in general skews towards readings that you find in English that are X-marked), but it is still compatible with the O-marked reading.
\ex. \gll 
ruguo Mali qu paidui, Yuehan jiu hui gaoxing. \\
if Mary go party John then will happy \\
\glt `If Mary goes to the party, John will be happy.'

\section{X-marked conditional}
\label{sec:x_marked_conditional}

X-marked counterfactual conditionals are usually identical to O-marked conditionals with the antecedent in perfective aspect.
\ex. \gll 
ruguo Mali zuotian qu-le paidui, Yuehan jiu hui gaoxing. \\
if Mary yesterday go-\textsc{pfv} party John then will happy \\
\a.[i.] `If Mary had come to the party, John would have been happy.'
\b.[ii.] `If Mary went to the party, John would be happy.'

However, the adverb \emph{benlai} `originally' in the consequent could force a counterfactual reading.
\ex. \gll 
ruguo Mali zuotian qu-le paidui, Yuehan \textbf{benlai} hui gaoxing de. \\
if Mary yesterday go-\textsc{pfv} party John originally will happy \textsc{de}\\
\a.[i.] `If Mary had come to the party, John would have been happy.'
\b.[ii.] *`If Mary went to the party, John would be happy.'

\emph{de} glossed \textsc{de} (here or below) is a possessive/modification/relativization marker. 
It has nothing to do with X-marking/tense/aspect, etc.

\section{Wish clauses}
\label{sec:wish_clauses}

\ex. \gll 
wo xiwang Mali lai paidui. \\
\textsc{1sg} hope Mary come party \\
\glt `I hope/wish Mary will come/would come to the party.'


The counterfactual wishes don't seem to be expressible with \emph{xiwang} or similar attitudes.
However, a conditional could help express it:
\ex. \gll 
yaoshi Mali lai-le paidui jiu hao le!\\
if.X-inclined Mary come-\textsc{pfv} party then good \textsc{sfp}  \\
\glt `If only Mary had come to the party!', literally, `If Mary had come to the party, then it would be good!'

\emph{le}, an \textsc{sfp} (sentence-final particle), usually encodes change-of-state.

\section{Correlatives}
\label{sec:correlatives}

\ex. \gll 
shei lai paidui, Yuehan jiu gei shei mai liwu. \\
who come party John then give who buy gift \\
\glt `Whoever comes to the party, John will buy a gift for them.'



\ex. \gll 
shei lai-le paidui, Yuehan jiu gei shei mai-le liwu. \\
who come-\textsc{pfv} party John then give who buy-\textsc{pfv} gift \\
\glt `Whoever came to the party, John bought a gift for them.'

\section{Unconditionals}
\label{sec:unconditionals}

\ex. \gll 
buguan shei lai paidui, Yuehan dou hui gaoxing. \\
no.matter who come party John even/all will happy \\
\glt `No matter who comes to the party, John will be happy.'

\section{Comparative correlatives}
\label{sec:comparative_correlatives}

No distinction between past and future.

\emph{yue} prefers to be in the clausal spine, and so a relative clause paraphrase is preferred.
\ex. \gll 
{\I[RC } lai paidui de] ren yue duo, Yuehan jiu yue gaoxing. \\
{} come party \textsc{de} person \textsc{yue} many John then \textsc{yue} happy \\
\a.[i.] `The more people will come to the pary, the happier John will be.'
\b.[ii.] `The more people came to the pary, the happier John were.'

\ex. \gll 
{\I[RC } lai paidui de] ren yue shao, Yuehan jiu yue gaoxing. \\
{} come party \textsc{de} person \textsc{yue} few John then \textsc{yue} happy \\
\a.[i.] `The fewer people will come to the pary, the happier John will be.'
\b.[ii.] `The fewer people came to the pary, the happier John were.'


\section{Comparative unconditionals}
\label{sec:comparative_unconditionals}

\ex. \gll 
buguan duoshao ren lai paidui, Yuehan dou hui dai zai fangjian li. \\
no.matter how.many person come party John even/all will stay be.at room in \\
\glt `No matter how many people will come to the party, John will stay in the room.'

\ex. \gll 
buguan duoshao ren lai-dao-le paidui, Yuehan dou dai zai-le fangjian li. \\
no.matter how.many person come-arrive-\textsc{pfv} party John even/all stay be.at-\textsc{pfv} room in \\
\glt `No matter how many people came to the party, John stayed in the room.'

With \emph{how few}, a relative clause is again preferred.
\ex. \gll 
buguan {\I[RC } lai paidui de] ren you duo(me) shao, Yuehan dou hui dai zai fangjian li. \\
no.matter {} come party \textsc{de} person exist how.much few John even/all will stay be.at room in \\
\glt `No matter how few the people who will come to the party will be, John will stay in the room.'

\ex. \gll 
buguan {\I[RC } lai paidui de] ren you duo(me) shao, Yuehan dou dai zai-le fangjian li. \\
no.matter {} come party \textsc{de} person exist how.much few John even/all stay be.at-\textsc{pfv} room in \\
\glt `No matter how few the people who come to the party were, John stayed in the room.'



\section{Comparatives}
\label{sec:comparatives}
In these cases, nominalization via relativization is strongly preferred.
\ex. \gll 
lai Yuehan de paidui de ren bi lai Mali de paidui de ren duo. \\
come John \textsc{de} party \textsc{de} person than come Mary \textsc{de} party \textsc{de} person many \\
\glt `More people came to John's party than Mary's party.'

\ex. \gll 
Yuehan de paidui bi Mali de paidui yongji. \\
John \textsc{de} party than Mary \textsc{de} party crowded \\
\glt `John's party was more crowded than Mary's party.'


\ex. \gll 
lai Yuehan de paidui de ren bi lai Mali de paidui de ren shao. \\
come John \textsc{de} party \textsc{de} person than come Mary \textsc{de} party \textsc{de} person few \\
\glt `Fewer people came to John's party than Mary's party.'

\ex. \gll 
Yuehan de paidui meiyou Mali de paidui yongji. \\
John \textsc{de} party \textsc{neg}.than Mary \textsc{de} party crowded \\
\glt `John's party was less crowded than Mary's party.'






\end{document}


