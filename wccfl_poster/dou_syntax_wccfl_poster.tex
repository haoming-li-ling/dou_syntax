% Unofficial UChicago CS Poster Template
% v1.1.0 released September 8, 2022
% https://github.com/k4rtik/uchicago-poster
% a fork of https://github.com/anishathalye/gemini

\documentclass[final]{beamer}

% ====================
% Packages
% ====================

% \usepackage[T1]{fontenc}
% \usepackage{lmodern}
\usepackage[width=48in,height=36in,scale=1.1]{beamerposter}
% \usepackage[size=custom,width=120,height=85,scale=1.1]{beamerposter}
\usepackage[dvipsnames]{xcolor}
\usetheme{gemini}
\usecolortheme{mit}

\usepackage{mathtools}
\usepackage{amsfonts, amssymb}
\usepackage{stmaryrd}
\usepackage{pifont}

% \usepackage{cleveref}
\usepackage{unicode-math}
\AtBeginDocument{\lingset{glstyle=wrap}}
\setmainfont{Fira Sans}
\setmathfont{Fira Math}
% \setmathfont{TeX Gyre DejaVu Math}[range=\bigoplus]
% % \usepackage{firamath-otf}
%
\setsansfont{Fira Sans}[
Ligatures=TeX,
UprightFont={* Regular},
  ItalicFont={* Italic},
  BoldFont={* Bold},
  BoldItalicFont={* Bold Italic}
]
% \setmathrm{Fira Sans}[
%   % UprightFont=*-Light,
%   BoldFont=*-Bold,
%   % ItalicFont=*-LightItalic,
%   % BoldItalicFont=*BoldItalic,
% ]
% \usepackage{float}
% \usepackage{newpx}
\usepackage{graphicx}
\usepackage{booktabs}
% \usepackage{framed}
\usepackage{adjustbox}
\usepackage{eqparbox}
\usepackage{doi}
\usepackage{setspace}
\renewenvironment{quote}
{\list{}{\rightmargin=1cm \leftmargin=1cm}\item\relax}{\endlist}
% \usepackage[numbers]{natbib}
\usepackage[backend=biber, natbib=true, style=unified]{biblatex}
\setbeamertemplate{bibliography item}{}
\setlength\bibitemsep{0pt}
% \AtBeginBibliography{\tiny}

% \AtBeginBibliography{\setstretch{0.1}}
\AtEveryBibitem{%
  \clearfield{pages}%
}
\renewcommand{\bibfont}{\normalfont\tiny}

% \defbibenvironment{bibliography}
%   {}
%   {}
%   {%\addspace
%    \printtext[labelnumberwidth]{%
%      \printfield{labelprefix}%
%      \printfield{labelnumber}}%
%    \addhighpenspace}

\defbibenvironment{bibliography}
  {
    % \noindent
  }
  {
    % \unspace
  }
  {
    \printtext[labelnumberwidth]{%
     \(\bullet\)}%
   \addspace
 }
\renewbibmacro*{finentry}{\finentry\addspace}
\addbibresource{../Chinese_Modals.bib}
\addbibresource{../Distributivity.bib}
\DeclareSourcemap{
  \maps[datatype=bibtex]{
    \map[overwrite]{
      \step[fieldsource=doi, null]
      \step[fieldset=url, null]
      \step[fieldset=urldate, null]
      \step[fieldset=doi, null]
      \step[fieldset=eprint, null]
      \step[fieldset=journaltitle, null]
      \step[fieldset=pages, null]
      \step[fieldset=volume, null]
      \step[fieldset=issue, null]
      \step[fieldset=number, null]
      \step[fieldset=institution, null]
      \step[fieldset=location, null]
      \step[fieldset=publisher, null]
      \step[fieldset=booktitle, null]
      \step[fieldset=eventtitle, null]
      \step[fieldset=editor, null]
      % \step[fieldset=date, year]
    }  
  }
}
% \DeclareSourcemap{
%   \maps[datatype=bibtex]{
%     \map[overwrite]{
%       \step[fieldsource=doi, final]
%       \step[fieldset=url, null]
%       \step[fieldset=doi, null]
%       \step[fieldset=eid, null]
%       \step[fieldset=eprint, null]
%       \step[fieldset=eprinttype, null]
%     }  
%   }
% }
% \urlstyle{tt}
\usepackage[patch=none]{microtype}
\usepackage{tikz}
\usetikzlibrary{positioning}
\usetikzlibrary{arrows.meta}
\usetikzlibrary{tikzmark}
\usepackage{tcolorbox}
\tcbuselibrary{skins}
\tcbuselibrary{raster}
\usepackage{pgfplots}
\pgfplotsset{compat=1.18}
\usepackage{anyfontsize}
\usepackage{multicol}
\usepackage{vwcol}

\makeatletter\def\new@fontshape{}\makeatother
% \usepackage{linguex, cgloss}

\usepackage[linguex]{expex-glossonly}


\AtBeginDocument{
  \setlength{\Exindent}{1ex}
  % \setlength{\Extopsep}{0\baselineskip}
  % \setlength{\Exredux}{0\baselineskip}
  % \setlength{\Exlabelwidth}{.4\Exlabelwidth}
  % \setlength{\Exlabelsep}{.4\Exlabelsep}
  % \setlength{\SubExleftmargin}{\SubExleftmargin}
  % \setlength{\SubSubExleftmargin}{\SubSubExleftmargin}
}
% \renewcommand\eachwordone{\textsf} %sans serif in glossed examples
% \renewcommand\eachwordtwo{\textsf}
\pdfstringdefDisableCommands{%
  \def\translate#1{#1}%
}

\usepackage{cleveref}
\crefname{ExNo}{}{}
\crefname{SubExNo}{}{}
\renewcommand{\theExNo}{\arabic{ExNo}}
\renewcommand{\theSubExNo}{\theExNo\alph{SubExNo}}
\creflabelformat{SubExNo}{(#2#1#3)}
\creflabelformat{ExNo}{(#2#1#3)}
\crefrangelabelformat{SubExNo}{(#3#1#4--#5\crefstripprefix{#1}{#2}#6)}
\crefrangelabelformat{ExNo}{(#3#1#4)--(#5#2#6)}
\usepackage{paracol}
\globalcounter{ExNo}


% ====================
% Lengths
% ====================

% If you have N columns, choose \sepwidth and \colwidth such that
% (N+1)*\sepwidth + N*\colwidth = \paperwidth
\newlength{\sepwidth}
\newlength{\colwidth}
\setlength{\sepwidth}{0.01\paperwidth}
\setlength{\colwidth}{0.32\paperwidth}

\newcommand{\separatorcolumn}{\begin{column}{\sepwidth}\end{column}}

\renewcommand<>{\emph}[1]{{\only#2{\em}#1}}
% ====================
% Title
% ====================

\title{Mandarin modals embed clauses: Evidence from focus-sensitive operators}

\author{Haoming Li}

\institute[shortinst]{
  Massachusetts Institute of Technology
}

% ====================
% Footer (optional)
% ====================

\footercontent{
  \href{https://haoming-li-ling.github.io}{haoming-li-ling.github.io} \hfill
  Sinn und Bedeutung 29 \hfill
  % LING 24.970 Introduction to Semantics \hfill
  \href{mailto:haomingl@mit.edu}{haomingl@mit.edu}
}
% (can be left out to remove footer)

% ====================
% Logo (optional)
% ====================

% use this to include logos on the left and/or right side of the header:
% \logoright{\includegraphics[height=7cm]{logo1.pdf}}
% \logoleft{\includegraphics[height=7cm]{logo2.pdf}}

% ====================
% Body
% ====================


\newcommand{\colorback}[2]{\mathchoice%
  {\colorbox{#1}{$\displaystyle#2$}}%
  {\colorbox{#1}{$\textstyle#2$}}%
  {\colorbox{#1}{$\scriptstyle#2$}}%
  {\colorbox{#1}{$\scriptscriptstyle#2$}}}%


\newcommand{\F}{\ensuremath{_{\mathrm{F}}}}
\newcommand{\opfs}{Op\(_{\mathrm{FS}}\)}
\newcommand{\topobj}{Topic\(_{\mathrm{int}}\)P}
\newcommand{\vmod}{V\(_{\mathrm{Mod}}\)}
\newcommand{\gap}[1]{\rule{1em}{0.4pt}\hspace{.1em}\textsubscript{#1}}

\begin{document}
\tcbset{
  % beamer, 
  enhanced, 
  % tile,
  bicolor,
  % sharp corners, 
  % enhanced,
  leftrule=-1pt,
  rightrule=-1pt,
  toprule=-1pt,
  bottomrule=-1pt,
  opacityfill=0,
  colframe=black,
  % titlerule=1pt, 
  toptitle=3pt,
  titlerule style={black},
  % colbacktitle=lightgray, 
  % colback=lightgray,
  colbacktitle=white, 
  colback=white,
  % colbacklower=red!10!white,
  colbacklower=lightgray,
  coltitle=mitred, 
  fonttitle=\bfseries, 
  raster column skip=10pt
  sidebyside align=top,
  halign=flush left
}

% \makeatletter
% \tcbset{
%   on layer/.code 2 args={%
%     \ifnum\c@tcblayer=\numexpr#1\relax
%       \pgfkeysalso{#2}%
%     \fi
%   }
% }
% \makeatother
%
% \tcbset{
%     on layer={2}{boxrule=3mm},
% }
\tcbset{every box on layer 2/.style={
    reset,
    halign=flush left
}}

\addtobeamertemplate{headline}{}
{
  \begin{tikzpicture}[remember picture,overlay]
    \node [anchor=north west, inner sep=3cm] at ([xshift=-1.2cm,yshift=-2.0cm]current page.north west)
      {\includegraphics[height=3.0cm]{mitlogos/MIT-logo-with-spelling-print-red-gray-design1.eps}}; % also try shield-white.eps
    % \node [anchor=north east, inner sep=3cm] at ([xshift=0.0cm,yshift=2.5cm]current page.north east)
    % {\includegraphics[height=8.0cm]{logos/cs-logo-white.png}};
  \end{tikzpicture}
}

\begin{frame}[t]
  \begin{columns}[t]
    \separatorcolumn

    \begin{column}{\colwidth}



      \begin{block}{Test 1: Locality of association with focus}
        \begin{itemize}

          \item \textbf{Modals pattern with CP-embedding verbs in blocking association with \opfs}.
          \item \opfs{} \emph{ye} `also' and \emph{dou} `even' allow backwards association:
            % \begin{tcolorbox}
              \ex. \begingl
                \gla  Lisi\F{} \textbf{dou} \nogloss{/} \textbf{ye} lai-le. //
                \glb  L. even also come-\textsc{pfv} //
                \glft `Lisi\F{} even/also came.' //
              \endgl

            % \end{tcolorbox}
          \item Assumption: \opfs{} covertly move to c-commanding the associate \cref{ex:foc_assoc}:
            \begin{tcolorbox}
              \ex. \label{ex:foc_assoc} \tikzmarknode{target}{\opfs} \ldots{} [ \ldots{} XP\F{} \ldots{} \tikzmarknode{base}{\gap{}} \ldots{} ] \hfill {\scriptsize \citep{lahiriFocusNegativePolarity1998,crnicNonmonotonicityNPILicensing2014,liuVarietiesAlternativesMandarin2017}}
              \begin{tikzpicture}[overlay, remember picture]
                \draw[-Stealth, line width=1mm] (base.south) -- ++(0, -.8) -| (target.south);
              \end{tikzpicture}
              \vspace{2ex}

            \end{tcolorbox}
          \item \opfs{} can move to very high positions in the clause, at least above TopicP:
            \begin{tcolorbox}
              \ex. \label{ex:base_topic} \begingl
                \gla Haixian\(_{\mathrm{F}}\), wo           \textbf{dou} zhi  xihuan chi zhu  shu    de          cai. //
                \glb fish                      \textsc{1sg} even         only like   eat cook cooked \textsc{de} dish //
                \glft `Even for seafood\F{}, I only like to eat cooked dishes.' //
              \endgl

            \end{tcolorbox}
          \item \emph{Yu} `fish' is a base-generated topic: no gap further down in the sentence.


          \item \opfs{} is unable to associate, and therefore by \cref{ex:foc_assoc}, move, across CP boundaries. 
            \begin{tcolorbox}
              \ex. \begingl
                \gla \ljudge{*}Lisi\F{} shuo \nogloss{\I{[}CP {} } @ Zhangsan \textbf{dou} lai-le @ \nogloss{]} //
                \glb L. say Z. even come-\textsc{pfv} //
                \glft Intended: `Lisi\F{} even said Zhangsan came.' //
              \endgl

            \end{tcolorbox}

          \item Then, \LLast and \Last jointly show that \opfs{} movement is subject to \Next:
            \begin{tcolorbox}[title={\opfs{} can freely move within a CP, but not out of it},colback=red!10!white]
              % \ex. \opfs{} movement is basically unrestricted by height in the smallest containing CP, but cannot escape it. \label{ex:loc_cons}

              \ex.  \tikzmarknode{target1}{\phantom{xxx}} \ldots{} \I{[}\textbf{CP} \tikzmarknode{target2}{\phantom{xxx}} \ldots{} \tikzmarknode{base1}{\opfs} \ldots{} ] 
              \begin{tikzpicture}[overlay, remember picture]
                \draw[-Stealth, line width=1mm] (base1.south) -- ++(0, -1.2)  -|  (target1.south) node[pos=.25,circle,fill=red!10!white,inner sep=-1pt]{\ding{55}};
                \draw[-Stealth, line width=1mm] (base1.south) -- ++(0, -.6)  -|  (target2.south) node[pos=.25,circle,fill=red!10!white,inner sep=-1pt]{\ding{51}};
              \end{tikzpicture}
              \vspace{3ex}



            \end{tcolorbox}
          \item \opfs{} are unable to associate across deontic modals:
            \begin{tcolorbox}[sidebyside, lefthand ratio=.42]
              \ex. \label{ex:fs_root}
              \begingl
                \gla Lisi\F{} \textbf{dou} \textbf{keyi} lai. //
                \glb L. even can come. //
                \glft `Lisi\F{} can also come.' //
              \endgl


              \tcblower
              \ex. \begingl
                \gla \ljudge{*}Lisi\F{} \textbf{keyi} \textbf{dou} lai. //
                \glb L. can even come //
                \glft Intended: `Lisi\F{} can also come.' //
              \endgl


            \end{tcolorbox}
          \item From the assumption in \cref{ex:foc_assoc}, this indicates that \opfs{} is unable to move across the modal.
          % \item Thus, deontic modals pattern with CP-embedding verbs in forbidding \opfs{} movement out of them.
          \item Without adding ad hoc modal-specific constraints, the simplest explanation is that modals are also CP-embedding verbs.
            \begin{tcolorbox}[colback=red!10!white]
              \ex.  \tikzmarknode{targett}{\opfs} \ldots{} [ \ldots{} XP\F{} \ldots{} \I{[}V Mod] \I{[}\textbf{CP} \ldots{} \tikzmarknode{basee}{\gap{}} \ldots{} ]] 
              \begin{tikzpicture}[overlay, remember picture]
                \draw[-Stealth, line width=1mm] (basee.south) -- ++(0, -.8)  -|  (targett.south) node[pos=.25,circle,fill=red!10!white,inner sep=0pt]{\ding{55}};
              \end{tikzpicture}
              \vspace{2ex}

            \end{tcolorbox}
          % \item This way, \cref{ex:loc_cons} can directly apply to derive the contrast in \LLast and \Last.
        \end{itemize}
      \end{block}
      \begin{block}{References}
        \setstretch{0.1}
        \printbibliography

      \end{block}


    \end{column}

    \separatorcolumn

    \begin{column}{\colwidth}
      \begin{alertblock}{Introduction}
        \begin{itemize}
          \item Long-standing debate in Mandarin: 
            \textbf{are modals clause-embedding lexical verbs (\emph{bi-clausal}, \Next), or higher heads in the same clause (\emph{mono-clausal}, \NNext)?}

          \item Both \Next and \NNext are intended to mean `Lisi can come.'
            \begin{tcolorbox}
              \ex. \begingl
                % \glpreamble  //
                \gla \rightcomment{\scriptsize \citet{linFinitenessClausesRaising2011,linMultiplemodalConstructionsMandarin2012,zhangSentencefinalAspectParticles2019}}Lisi \nogloss{\I{[}VP {} } @  \nogloss{\I{[}V {} } @ keyi @ \nogloss{]} \nogloss{\I{[}TP/CP {} } @ \nogloss{\gap{}} lai @ \nogloss{]]} //
                \glb L. can come //
              \endgl

              \ex.  \begingl
                % \glpreamble  //
                \gla \rightcomment{\scriptsize \citet{tsaiTopographyChineseModals2015,yipModalMovementLicensed2022,erlewineLowSentencefinalParticles2017}}Lisi \nogloss{\I{[}ModP \I{[}Mod {} } @ keyi @ \nogloss{]} \nogloss{\I{[}\emph{v}P {} } @ \nogloss{\gap{}} lai @ \nogloss{]]} //
                \glb L. can come //
              \endgl

            \end{tcolorbox}

          \item \citet{chappellVariationGrammaticalizationComplementizers2008,huangControlComplementsMandarin2018}: \emph{shuo} is a complementizer, strong evidence that deontic modals are clause(CP)-embedding:
            \begin{tcolorbox}
              \ex. \begingl
                \gla Lisi keyi \textbf{shuo} xian shi-le zai zuo jueding. //
                \glb L. can \textsc{comp} first try-\textsc{pfv} then make decision //
                \glft `Lisi can first try it and then make a decision.' //
              \endgl

            \end{tcolorbox}

           \begin{tcolorbox}[title=The main takeaway,colback=orange!20!white]
              \textbf{Three diagnostic tests based on focus-sensitive operators} (\opfs{}) \emph{dou} and \emph{ye} that further show that \textbf{deontic modals are clause/CP-embedding} even in the absence of \emph{shuo}.
            \end{tcolorbox}

            \begin{tcolorbox}[title=Theoretical implication: Implicational Complementation Hierarchy]
            \citet{wurmbrandImplicationalUniversalComplementation2020}: semantic clause-types (\emph{Event}, \emph{Situation}, \emph{Proposition}) have increasing \emph{minimum sizes} (\emph{v}P, TP, CP), but not size \emph{upper bounds}.
            \end{tcolorbox}

        \end{itemize}



      \end{alertblock}
      % \tcbset{colback=red!10!white,colbacklower=lightorange}

      \begin{block}{Test 2: Scope of focus-sensitive operators}
        \begin{itemize}
          \item \textbf{Presuppositions of \opfs{} project universally and epistemically from the scope of modals, like CP-taking verbs.}
          \item \Next presupposes the epistemic statement that Lisi is \textbf{expected} to write something else tomorrow; it is infelicitous where Lisi is only \textbf{allowed} to write something else.
            % which is correctly predicted to be grammatical by Test 1 (both the \opfs{} and its associate are on the same side of the modal; \cref{ex:loc_cons} is not violated).
            \begin{tcolorbox}\ex. \begingl
                \gla Lisi keyi lunwen\F{} \textbf{ye} mingtian xie. //
                \glb L. can paper also tomorrow write //
                \glft `Lisi is allowed to also write the paper\F{} tomorrow.' //
              \endgl

            \end{tcolorbox}
          \item \ldots just like \Next,  where it is presupposed that Lisi \textbf{believes} someone other than Zhangsan will come; it is infelicitous where Lisi only \textbf{hopes} someone other than Zhangsan will come.
            \begin{tcolorbox}
              \ex. \begingl
                \gla Lisi xiwang Zhangsan\F{} \textbf{ye} lai. //
                \glb L. hope Z. also come //
                \glft `Lisi hopes that Zhangsan\F{} will also come.' //
              \endgl

            \end{tcolorbox}
          % \item Example \LLast necessarily presupposes the epistemic statement that Lisi is \emph{expected} to write something else tomorrow.
          \item The presupposition is derived if we assume that 1) \emph{ye} is interpreted in an embedded position below \emph{keyi}, and that 2) presuppositions project universally and epistemically from under deontic modals à la \citet{heimPresuppositionProjectionSemantics1992}.
          \item \LLast is unlike the English mono-clausal modal construction
            \begin{tcolorbox}
              \ex. Lisi can also write their paper\F{} tomorrow.

            \end{tcolorbox}
          \item \Last only has the deontic presupposition that Lisi is \textbf{allowed} to write something else tomorrow, not that Lisi is \textbf{expected} to do so.
          % \item The reading where \emph{ye} outscopes \emph{keyi}, merely presupposing that Lisi is \emph{allowed} to write something else tomorrow, is unavailable.
          % \item \LLast is infelicitous where only the deontic presupposition is met.
            % \item Again, modalized sentences pattern with clause-embedding verbs \Last.
          % \item \emph{Ye} must be interpreted under \emph{xiwang} in \Last.

        \end{itemize}

      \end{block}

    \end{column}

    \separatorcolumn

    \begin{column}{\colwidth}

      \tcbset{colbacklower=red!10!white}

      \begin{block}{Test 3: Distribution of focus-sensitive operators}
        \begin{itemize}
          \item \textbf{Modals allow \opfs{} to precede items that they otherwise cannot in the same clause.}
          \item \emph{Ye} `also' and \emph{dou} `even,'  always pre-verbal, cannot precede fronted objects (TP-internal topics in Spec,\topobj,  \citealt{chenObscuredUniversalityMandarin2023}).
          % \item Both \Next and \NNext mean `Lisi\F{} also wrote the paper.'
            \begin{tcolorbox}
              \ex. \begingl
                \gla Lisi\F{} \nogloss{\{} @ \textbf{lunwen}\(_1\) \textbf{ye} @ \nogloss{/} \nogloss{*} @ \textbf{ye} \textbf{lunwen}\(_1\) @ \nogloss{\}}  xie-le \nogloss{\gap{1}}. //
                \glb L. paper also also paper write-\textsc{pfv} //
                \glft `Lisi\F{} also wrote the paper.' //
              \endgl

              \tcblower
              \ex. \label{ex:order_top_fs} \topobj{} \(\prec\) \opfs{} \(\prec\) V
              % \ex. \begingl
              %   \gla \ljudge{*}Lisi\F{} \textbf{ye} \textbf{lunwen}\(_1\) xie-le \nogloss{\gap{1}}. //
              %   \glb L. also paper write-\textsc{pfv} //
              %   \glft Intended: `Lisi\F{}  did his paper' //
              % \endgl
              %

            \end{tcolorbox}
          % \item Spec,\topobj{} must be higher than \opfs{} at base:
          %   \begin{tcolorbox}[colback=red!10!white]
          %     \ex. \label{ex:order_top_fs} \topobj{} \(\prec\) \opfs{} \(\prec\) V
          %
          %   \end{tcolorbox}
          \item When there are modals, \opfs{} can precede fronted objects:
          % \item Ignoring the modal, \Next has the sequence in \NNext:
            \begin{tcolorbox}
              \ex.
              \begingl
% \glpreamble Fronted objects //
                \gla Lisi\F{} \textbf{ye} \textbf{keyi} \textbf{lunwen}\(_1\) mingtian xie \nogloss{\gap{1}}. //
                \glb L. also can paper tomorrow write //
                \glft `Lisi\F{} is also allowed to write the paper tomorrow.' //
              \endgl

              \tcblower
              \ex. \opfs{} \(\prec\) \topobj{} \(\prec\) V  \label{ex:paradox}

            \end{tcolorbox}

          \item The sequence in \Last is in conflict with \cref{ex:order_top_fs}.
          \item \textbf{Explanation}: modals embed clauses at least \topobj{}, or for simplicity, CPs:

          % \item The sequence in \Last can then be rewritten as \Next:
            \begin{tcolorbox}[title=\cref{ex:order_top_fs} and \cref{ex:paradox} reconciled via two separate clausal ordering domains,colback=red!10!white]
              \ex. \I{[}CP \opfs{} \(\prec\) \vmod{} \(\prec\) \I{[}CP \topobj{} \(\prec\) V  ]]

            \end{tcolorbox}

          \item In each CP in \Last, \cref{ex:order_top_fs} is not violated.
          \item \textbf{Correct prediction}: when \opfs{} and \topobj{} are not separated by modal-induced clausal boundaries, they obey \cref{ex:order_top_fs}.
          % \item If the object is fronted across the modal, it precedes \opfs{}:
            \begin{tcolorbox}[title=Both \topobj{} and \opfs{} outside modal scope]
              \ex. \begingl
                \gla Lisi\F{} \nogloss{\{} @  \textbf{lunwen}\(_1\) \textbf{ye} \nogloss{/} \nogloss{*} @ \textbf{ye} \textbf{lunwen}\(_1\)  @ \nogloss{\}} keyi mingtian xie \nogloss{\gap{1}}. //
                \glb L. paper also also paper can tomorrow write //
                \glft `Lisi\F{} is also allowed write the paper tomorrow.' //
              \endgl

              \tcblower

              \ex. \I{[}CP \topobj{} \(\prec\) \opfs{} \(\prec\) \vmod{} \(\prec\) \I{[}CP V  ]] \label{ex:order_matrix}

            \end{tcolorbox}

          % \item Second, if \topobj{} and \opfs{} are both in the scope of the modal, the object also precedes \opfs{}:
            \begin{tcolorbox}[title=Both \topobj{} and \opfs{} inside modal scope]
              \ex.
              \begingl
                \gla Lisi keyi \nogloss{\{} @  \textbf{lunwen}\(_{1, \mathrm{F}}\) \textbf{ye} @ \nogloss{/} \nogloss{*} @ \textbf{ye} \textbf{lunwen}\(_{1, \mathrm{F}}\) @ \nogloss{\}} mingtian xie \nogloss{\gap{1}}. //
                \glb L. can  paper also also  paper tomorrow write //
                \glft `Lisi is allowed to also write the paper\F{} tomorrow.' //
              \endgl

              \tcblower
              \ex. \I{[}CP \vmod{} \(\prec\) \I{[}CP \topobj{} \(\prec\) \opfs{} \(\prec\) V  ]] \label{ex:order_embedded}

            \end{tcolorbox}

          % \item Clearly, both \cref{ex:order_matrix} and \cref{ex:order_embedded} follow \cref{ex:order_top_fs}.
        \item This means that modals essentially `reset' the clausal projections and therefore the sequence that \opfs{} and \topobj{} follow.\end{itemize}
        \end{block}
        
      \end{column}
      \separatorcolumn
    \end{columns}
\end{frame}




\end{document}
